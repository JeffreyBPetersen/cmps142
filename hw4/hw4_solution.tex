\documentclass{article}
\usepackage{amsmath, amssymb, graphicx, verbatim}
\usepackage[margin=1in]{geometry}
\title{CMPS 142: Homework Assignment 4}
\author{Jeffrey Petersen - 1329242\\Peter Czupil - 1317993\\Raymond Colebaugh - 1377877}
\begin{document}
\maketitle
\begin{enumerate}
        \item 
            \begin{enumerate}
                \item
                    \noindent{The probability that a point randomly drawn from $p$ is located somewhere in the interval $(z_{\epsilon}, \theta)$ is equal to $\epsilon$. Thus the probability that a point falls outside this interval is the complement of the previous probability. Therefore, $p((0, z_{\epsilon}]) = 1 - \epsilon$. }
                \item
                    Assuming that the training set may contain duplicate $x$ values, the probability that all points lie outside the interval $(z_{\epsilon}, \theta]$ is the product of the probability from part (a) for all $x$'s in the training set. Therefore $p(X \notin (z_{\epsilon}, \theta]) = \prod_{i = 1}^{N} 1 - \epsilon = (1 - \epsilon)^N$ 
                \item
                    Since, $p((z_{\epsilon}, \theta])$ is equal to $\epsilon$, the probability that ${\hat {\theta}}$ has an error rate of at least $\epsilon$ is the probability that ${\hat {\theta}} \leq z_{\epsilon}$. This is the area under our density function, on the interval $[0, z_{\epsilon}]$. We had computed this area previously in part (b) which turned out to be equal to $(1 - \epsilon)^N$ where N is the cardinality of the training set. Therefore the probability that ${\hat {\theta}}$ has an error rate of at least $\epsilon$ is equal to $(1 - \epsilon)^N$.
                \item
                 To find the smallest $N$ s.t. $\epsilon  \leq error\ rate\ of\ {\hat {\theta}} \leq \delta$, we can take our result from (c) and use the following inequality: 
$$ (1 - \epsilon)^N \leq \delta $$
Taking the log of both sides gives us:
$$Nlog(1 - \epsilon) \leq log(\delta)$$ 
which implies:
$$N \geq \frac{log(\delta)}{log(1 - \epsilon)}$$
            \end{enumerate}
\end{enumerate}
\end{document}
