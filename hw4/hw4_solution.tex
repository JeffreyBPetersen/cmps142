\documentclass{article}
\usepackage{amsmath, amssymb, graphicx, verbatim}
\usepackage[margin=1in]{geometry}
\title{CMPS 142: Homework Assignment 4}
\author{Jeffrey Petersen - 1329242\\Peter Czupil - 1317993\\Raymond Colebaugh - 1377877}
\begin{document}
\maketitle
\begin{enumerate}
        \item 
            \begin{enumerate}
                \item
                    \noindent{The probability that a point randomly drawn from $p$ is located somewhere in the interval $(z_{\epsilon}, \theta)$ is equal to $\epsilon$. Thus the probability that a point falls outside this interval is the complement of the previous probability. Therefore, $p((0, z_{\epsilon}]) = 1 - \epsilon$. }
                \item
                    Assuming that the training set may contain duplicate $x$ values, the probability that all points lie outside the interval $(z_{\epsilon}, \theta]$ is the product of the probability from part (a) for all $x$'s in the training set. Therefore $p(X \notin (z_{\epsilon}, \theta]) = \prod_{i = 1}^{N} 1 - \epsilon = (1 - \epsilon)^N$ 
                \item
            \end{enumerate}
        \item 
\end{enumerate}
\end{document}
