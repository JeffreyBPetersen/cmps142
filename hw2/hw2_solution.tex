\documentclass{article}
\usepackage{amsmath, amssymb}
\title{CMPS 142: Homework Assignment 2}
\author{Raymond "The Dude" Colebaugh - 1377877\\Jeffery "\#YOLOSWAG" Petersen - ID\\Peter "Gettin' there man" Czupil - 1317993}
\begin{document}
\maketitle
\begin{enumerate}
        \item 
            \begin{enumerate}
                \item
                    \begin{tabular}{l | c c c c c c}
                        Algorithm      & Correct & Incorrect & MAE & RMS & RAE (\%) & RSE (\%)\\
                        \hline
                        Nearest Neighbor (IB1) & 768 & 0   & 0      & 0      & 0 & 0 \\
                        Naive Bayes            & 586 & 182 & 0.2811 & 0.4133 & 61.8486 & 86.7082 \\
                        Logistic Regression    & 601 & 167 & 0.3063 & 0.3908 & 67.3928 & 81.9907
                    \end{tabular}
                \item 
                \item
                    \begin{tabular}{l | c c c c c c}
                        Algorithm      & Correct & Incorrect & MAE & RMS & RAE (\%) & RSE (\%)\\
                        \hline
                        Nearest Neighbor (IB1) & 539 & 229 & 0.2982 & 0.5461 & 65.6046 & 114.5627 \\
                        Naive Bayes            & 586 & 182 & 0.2841 & 0.4168 & 62.5028 & 87.4349 \\
                        Logistic Regression    & 593 & 175 & 0.3094 & 0.3954 & 68.0819 & 82.9651
                    \end{tabular}
                \item
                \item
                \item
            \end{enumerate}
        \item
            \begin{enumerate}
                \item The outcome space is the set of possible combinations of whether
                      each of the two children were male or female. Given in the notation
                      of $(younger, older)$, this leads to the outcome space:
                      $$
                        (B, G), (G, B), (B, B), (G, G)
                      $$
                \item We want the probability of at least one child being a girl, given
                      that we already know one child is a boy. We reduce the outcome space
                      by removing the possiblity of $(G, G)$. Then the remaining outcomes
                      are $(B, G), (G, B), (B, B)$. This leaves us with a probability
                      of $\frac{2}{3}$.
                \item Given that we know one child is a boy, then we are left with two
                    cases: either it was the first or the second child.  \\
                    \begin{tabular}{l | r}
                        \begin{tabular}{c c}
                            B & B \\
                            B & G \\
                            -G & B \\
                            -G & G
                        \end{tabular}
                        &
                        \begin{tabular}{c c}
                            B & B \\
                            -B & G \\
                            G & B \\
                            -G & G
                        \end{tabular}
                    \end{tabular} \\
                    This results in two cases where one child is a girl, out of four
                    remaining cases, for a probability of $\frac{1}{2}$.
            \end{enumerate}
        \item
            Given the existing data, we can calculate the mean of the GPA of
            honors students to be $ \frac{4.0 + 3.7 + 2.5}{3} = 3.4 $.
            $$ P( X_{GPA} ) = \frac{1}{\sqrt{2 \pi} 0.6} e^{\frac{(x - 3.4)^2 }{1.2} }$$
            $$ P( X_{AP} ) = \prod\limits_{i = 1}^{n} {P_H}^{x^{(i)}}(1 - P_H)^{1 - x^{(i)}} $$
            Our final prediction is: \\
            If AP courses are taken, predict $H$ if the GPA is between ..., and
            if AP courses are not taken, predict $H$ if the GPA is between ... \\
        \item
            $$ E[V]E[W] = E[VW] $$
            $$ E[V] = \sum_{i = 1}^{n} v_i P(v_i), E[W] = \sum_{j = 1}^{n} w_i P(w_i) $$
            $$ E[VW] = \sum_{i = 1}^{n} v_iw_iP(v_i, w_i) = \sum_{i = 1}^{n} v_iP(v_i)w_iP(v_i)$$
            $$ = \sum_{i = 1}^{n} v_i P(v_i) \sum_{i = 1}^{n} w_i P(w_i) $$
            $$ = E[V]E[W] $$
\end{enumerate}
\end{document}
